%%%%%%%%%%%%%%%%%%%%%%%%%%%%%%%%%%%%%%%%%
% Medium Length Professional CV
% LaTeX Template
% Version 2.0 (8/5/13)
%
% This template has been downloaded from:
% http://www.LaTeXTemplates.com
%
% Original author:
% Rishi Shah 
%
% Important note:
% This template requires the resume.cls file to be in the same directory as the
% .tex file. The resume.cls file provides the resume style used for structuring the
% document.
%
%%%%%%%%%%%%%%%%%%%%%%%%%%%%%%%%%%%%%%%%%

%----------------------------------------------------------------------------------------
%	PACKAGES AND OTHER DOCUMENT CONFIGURATIONS
%----------------------------------------------------------------------------------------

\documentclass{resume} % Use the custom resume.cls style
\usepackage{enumitem}
\usepackage[left=0.75in,top=0.6in,right=0.75in,bottom=0.6in]{geometry} % Document margins
\usepackage{hyperref}
\hypersetup{
	colorlinks=true,
	linkcolor=blue,
	filecolor=blue,      
	urlcolor=blue,
	citecolor=cyan,
}
\newcommand{\tab}[1]{\hspace{.2667\textwidth}\rlap{#1}}
\newcommand{\itab}[1]{\hspace{0em}\rlap{#1}}
\name{Jianwen Song} % Your name

\address{Homepage: https://jianwensong.github.io/}
\address{Address: No.24 South Section 1, Yihuan Road, Chengdu , China, 610065} % Your address
%\address{123 Pleasant Lane \\ City, State 12345} % Your secondary addess (optional)
\address{Email: jianwensong6@gmail.com} % Your phone number and email  Phone: (+86)155xxxxxxxx \\

\begin{document}

%----------------------------------------------------------------------------------------
%	EDUCATION SECTION
%----------------------------------------------------------------------------------------
\begin{rSection}{Education}
	{\bf Sichuan University} \hfill {\em Sep. 2017 - Jun. 2020} 
	\\ Master of Engineering, Detection Technology and Automatic Equipment\\ GPA: \textbf{91/100 (3.8/4.0)}\\
	Supervisor: Kai Liu\\
	\\{\bf Sichuan University} \hfill {\em Sep. 2013 - Jun. 2017} 
	\\ Bachelor of Engineering, Electronics and Information Engineering\\GPA: 85.42/100 (3.38/4.0)\\
	Supervisor: Wei Wu
	%Minor in Linguistics \smallskip \\
	%Member of Eta Kappa Nu \\
	%Member of Upsilon Pi Epsilon \\
\end{rSection}

\begin{rSection}{Research interests}
3D vision; Super-resolution; Point cloud processing
\end{rSection}
%--------------------------------------------------------------------------------
%    Projects And Seminars
%-----------------------------------------------------------------------------------------------
\begin{rSection}{Publications}

\begin{itemize}[leftmargin=*]
\item Kai Liu, Kangkang Zhang, Jinghe Wei, {\bf Jianwen Song}, Daniel L. Lau, Ce Zhu and Bin Xu*, ``Extending epipolar geometry for real-time structured light illumination", {\bf Optics Letters}, 2020, 45(12): 3280-3283. \href{https://www.osapublishing.org/ol/abstract.cfm?uri=ol-45-12-3280}{[Paper]}
\item Kai Liu, Wenqi Hua, Jinghe Wei, {\bf Jianwen Song}, Daniel L. Lau, Ce Zhu and Bin Xu*, ``Divide and conquer: High-accuracy and real-time 3D reconstruction of static objects using multiple-phase-shifted structured light illumination", {\bf Optics Express}, 2020, 28(5): 6995-7007. \href{https://www.osapublishing.org/oe/abstract.cfm?uri=oe-28-5-6995}{[Paper]}
\item Kai Liu, {\bf Jianwen Song}, Daniel L. Lau, Xiujuan Zheng, Ce Zhu, and Xiaomei Yang*, ``Reconstructing 3D point clouds in real time with look-up tables for structured light scanning still objects along both horizontal and vertical directions", {\bf Optics Letters}, 2019, 44(4): 6029-6032. \href{https://www.osapublishing.org/ol/abstract.cfm?uri=ol-44-24-6029}{[Paper]}
\item {\bf Jianwen Song}, Daniel L. Lau, Yo-Sung Ho, and Kai Liu*, ``Automatic look-up table based real-time phase unwrapping for phase measuring profilometry and optimal reference frequency selection", {\bf Optics Express}, 2019, 27(9): 13357-13371. \href{https://www.osapublishing.org/oe/abstract.cfm?uri=oe-27-9-13357}{[Paper]}
\item {\bf Jianwen Song}, Yo-Sung Ho, Daniel L. Lau, and Kai Liu*, ``Universal phase unwrapping for phase measuring profilometry using geometric analysis", Proc. SPIE (Emerging Digital Micromirror Device Based Systems and Applications X), 2018, 10546: 105460B. \href{https://www.spiedigitallibrary.org/conference-proceedings-of-spie/10546/105460B/Universal-phase-unwrapping-for-phase-measuring-profilometry-using-geometry-analysis/10.1117/12.2289423.short?SSO=1}{[Paper]}
\item Zhenli, Xiaomin Yang, {\bf Jianwen Song}, Kai Liu, Zuping Wang, and Wei Wu*, ``Improving Resolution of 3D Surface With Convolutional Neural Networks", Sustainable Cities and Society, 2018, 42: 127-138. \href{https://www.sciencedirect.com/science/article/pii/S2210670718303676}{[Paper]}
\item {\bf Jianwen Song}, Daniel L. Lau, Xiaomin Yang, Bin Xu, and Kai Liu*, ``Universal decoding method for periodic patterns in phase shifting structured light illumination", Optics Express. [Preparing]
\end{itemize}
\end{rSection}

\begin{rSection}{Patents}
\begin{itemize}[leftmargin=*]
	\item Kai Liu, {\bf Jianwen Song}, Jiang Wang, and Yiguang Liu, ``Customized projector and projection method based on one-dimensional information", Chinese Patent, CN105737761B.
	\item Kai Liu, {\bf Jianwen Song}, Ziyang Hu, and Bin Xu, ``Phase unwrapping method, device and electronic instrument based on two-dimensional look-up table ", Chinese Patent, CN110006365B.
	\item Kai Liu, {\bf Jianwen Song}, and Han Zhang, ``System calibration method, device, and three-dimensional reconstruction system ", Chinese Patent, CN107170010A. [Patent pending]
	\item Kai Liu, {\bf Jianwen Song}, Jun Gong, and Ce Zhu, ``Three-dimensional reconstruction method and system based on structured light periodical pattern", Chinese Patent, CN110285775A. [Patent pending]
\end{itemize}
	
\end{rSection}

%----------------------------------------------------------------------------------------
%	WORK EXPERIENCE SECTION
%----------------------------------------------------------------------------------------

\begin{rSection}{Projects}
	\begin{itemize}[leftmargin=*]
		\item Fast and high-precision 3D shape measurement system using surface structure light, supported by Chengdu Zhongliang Electronic Technology Corporation \hfill{\em Jun. 2019}\vspace{1.5mm}\\	
		-- Build the whole system and design the operation interface programmed using C++.
		
		\item  Low-cost and fast structured light 3D imaging for product examination in high-end manufacturing, supported by Science and Technology Support Program of Chengdu City 
		\hfill {\em Sep. 2018 - Mar. 2020}\vspace{1.5mm}\\
		-- Study different kinds of coding schemes in structured light illumination and compare their accuracy and efficiency.
		
		\item  High-precision structured light 3D imaging for pressure vessel and pipeline deformation corrosion detection, supported by Science and Technology Support of Sichuan University and Zigong City Cooperation Project \hfill {\em Sep. 2018 - Jan. 2020}\vspace{1.5mm}\\
		-- Design a high-accuracy and efficient phase unwrapping algorithm for structured light illumination.
		
		\item  Inexpensive real-time high-precision multi-frame structured light 3D face imaging, supported by Beijing Baidu Network Technology Corporation 
		\hfill {\em May. 2018 - Apr. 2019}\vspace{1.5mm}\\
		-- Design a universal decoding method that can be applied to different coding schemes of structured light illumination.
				
		\item  High-precision structured light 3D imaging for high-reflective aero-engine blades, supported by Science and Technology Support Program of Sichuan Province 
		\hfill {\em Jan. 2018 - May. 2020}\vspace{1.5mm}\\
		-- Design an algorithm that can detect the light saturation area in a structured light system.
		
		\item  High-precision structured light 3D imaging for illumination saturation overflow and multipath effects, supported by the National Natural Science Foundation of China 
		\hfill {\em Sep. 2017 - Dec. 2018}\vspace{1.5mm}\\		
		-- Study the influence of point spread function on a structured light system.
		
		\item  Pipeline analysis system, supported by Seikowave Corporation 
		\hfill {\em Jun. 2017 - Sep. 2018}\vspace{1.5mm}\\		
		-- Fit the rotation parameter of given pipeline point clouds by using Hough transform.
							
	\end{itemize}
\end{rSection}

\begin{rSection}{Honors and Awards}
	\begin{itemize}[leftmargin=*]
		\item \textbf{National Scholarship for Postgraduate Student}\hfill {\em Nov. 2018 and Nov. 2019}
		\item Outstanding Postgraduate Student of Sichuan University \hfill {\em Nov. 2018 and Oct. 2019} 
		\item First-class Postgraduate Academic Scholarship of Sichuan University \hfill {\em Sep. 2017-Jun. 2020}
		\item Outstanding Undergraduate Thesis of Sichuan University \hfill {\em Jun. 2017}
		\item Outstanding Undergraduate Student of Sichuan University \hfill {\em Oct. 2016 and Oct. 2014} 
		\item Third-class Scholarship of Sichuan University \hfill {\em Oct. 2016 and Oct. 2014}
		\item {1$^{\rm st}$} Prize, Ship Model Competition of Sichuan University \hfill {\em Apr. 2014}
	\end{itemize}
	
\end{rSection}

\begin{rSection}{Skills}
	\item Programming Languages: MATLAB, C/C++, Python
	\item Toolkits: \LaTeX, OpenCV, PCL, PyTorch, TensorFlow, Linux, etc.
	\item English Proficiency: IELTS (Overall 7.0, Reading 8.0, Listening 8.0, Writing 6.5, Speaking 6.0)
\end{rSection}
\end{document}
